\documentclass[journal, onecolumn, 11pt]{IEEEtran}

\usepackage{lineno}
%\usepackage{appendix}
\usepackage{amsmath}
\usepackage{xcolor}
\usepackage{framed}
\usepackage{graphicx}
\usepackage{tikz}
\usepackage{color}
\usepackage{multirow}
\usetikzlibrary{shapes,arrows}
\usepackage{subcaption}
\usepackage{algorithm}
\usepackage[normalem]{ulem}
\usepackage[algo2e]{algorithm2e}

%\usepackage[bookmarks=false]{hyperref}
%\usepackage[noend]{algpseudocode}

\usepackage{afterpage}

\newcommand\blankpage{%
	\null
	\thispagestyle{empty}%
	\addtocounter{page}{-1}%
	\newpage}
\usepackage{enumitem,amssymb}
\newlist{todolist}{itemize}{2}
\setlist[todolist]{label=$\square$}
\newcommand{\norm}[1]{\left\lVert#1\right\rVert}
\newcommand{\Tr}[1]{\text{Tr}\left(#1\right) }
\newcommand{\vt}[1]{\text{vec}\left(#1\right) }
\newcommand*{\tran}{^{\mkern-1.5mu\mathsf{T}}}

\newcommand{\Logm}[1]{\text{Logm}\left(#1\right) }
\newcommand{\Expm}[1]{\text{Expm}\left(#1\right) }

%\modulolinenumbers[5]

\newcommand\blfootnote[1]{%
	\begingroup
	\renewcommand\thefootnote{}\footnote{#1}%
	\addtocounter{footnote}{-1}%
	\endgroup
}
\colorlet{shadecolor}{blue!20}
\newcommand{\mgw}[1]{\textcolor{orange}{#1}}
\newcommand{\vj}[1]{\textcolor{blue}{#1}}
\newcommand{\jxu}[2]{\textcolor{olive}{\sout{#1}} \textcolor{red}{#2}}

\usepackage[style=numeric, sorting=none, giveninits=true, backend=bibtex, maxbibnames=3, url=false, isbn=false, doi=false]{biblatex}
%style=numeric-comp ,
\bibliography{library}

\title{Checklist for personalized NIBS for L2 learning}

\makeatletter
\def\blx@maxline{77}
\makeatother


\nolinenumbers
\begin{document}
	
	%\begin{itemize}
	%	\item Riemannian methods suffer from poor interpretability and high computational cost
	%	\item Spatial filters from the tangent space enable interpretable and efficient computation
	%	\item Applying the filters are equivalent to the original formulation under mild conditions
	%	\item Proposed pipelines meet/increase the original method's accuracy in a meta-analysis
	%	\item We include open-source off-the-shelf code for the proposed framework
	%\end{itemize}
	
	\numberwithin{equation}{section}
	\numberwithin{figure}{section}
	\numberwithin{table}{section}
	
	\begin{center}
		{ \Large  
			\textbf{Checklist for personalized NIBS for L2 learning - Electrodes placement}
		}
	\end{center}
\vspace{1cm}

\begin{itemize}
	\item \textbf{Montage}
	
	\begin{todolist}
		\item FC3: Anode (Red) \& T7: Cathode (Blue)
		\item Stim. elec. Vicinity: FFC3h, FFC5h, FCC3h, FCC5h \& None
	\end{todolist}
	\item \textbf{Preparation tACS Electrodes}
	\begin{todolist}
		\item \textbf{EEG cap}: Measure brain size (circumference) $\Rightarrow$ Choose and put on EEG cap $\Rightarrow$ Adjust position (10-20 system) $\Rightarrow$ \textbf{Fasten the chin strap}
		\item \textbf{Location marking}: Use EOG sticker to mark FC3 \& T7 $\Rightarrow$ Remove the caps $\Rightarrow$ Remove the hair in marked area 

		\item \textbf{Stimulator:} OFF $\Rightarrow$  Connect the wire $\Rightarrow$  Apply Ten20 gel on the tACS electrodes $\Rightarrow$ Place it in the marked area $\Rightarrow$  Turn ON stimulator

		\item \textbf{Impedance check (10k Ohm)}: Add more paste or press until the impedance is under 10k$\Omega$ and carefully pressing on the tACS electrodes to avoid tACS electrodes sliding. 
		\item\textbf{Remove the excess gel with a cotton swab}
		\item Turn OFF the stimulator
	\end{todolist}
	\item \textbf{Mounting the EEG cap}
		\begin{todolist}
			\item Carefully and gently put on the cap and MAKE SURE the tACS electrodes are not shifted and no Ten20 gel leaked out
			\item Fasten the strap of the EEG cap
		\end{todolist}
	
	\item \textbf{Preparation for EEG electrodes}
		\begin{todolist}
			\item\textbf{Gel electrode:} ground\&ref   $\Rightarrow$ Vicinity of the tACS electrodes $\Rightarrow$ Rest
			\item \textbf{Note for vicinity}: Inject the gel with the needle tip pointing in a direction away from the tACS electrode and apply as LESS GEL as possible.
			\item First round
			\item Second round: In the second round use the \textbf{cotton swab for the vicinity}
			\item \textbf{EEG impedance check}: gel until green/yellow
			\item \textbf{!!! Hardware check:} DC mode + 1Kz sampling frequency
			\item \textbf{!!! Stimulator setting:} Signal out ON, REMOTE ON
			\item \textbf{Saturation test}, i.e., apply 1mA sinusoidial signal and observe any saturation of EEG electrodes from recording signal. (safe\_start.py \& safe\_end.py)
			\item \textbf{Check EEG AND tACS impedance again and start recording.}
			
		\end{todolist}
\end{itemize}
\blankpage

\blankpage
	
	
	
	
	
\end{document}
