\documentclass[journal, onecolumn, 11pt]{IEEEtran}

\usepackage{lineno}
%\usepackage{appendix}
\usepackage{amsmath}
\usepackage{xcolor}
\usepackage{framed}
\usepackage{graphicx}
\usepackage{tikz}
\usepackage{color}
\usepackage{multirow}
\usetikzlibrary{shapes,arrows}
\usepackage{subcaption}
\usepackage{algorithm}
\usepackage[normalem]{ulem}
\usepackage[algo2e]{algorithm2e}

%\usepackage[bookmarks=false]{hyperref}
%\usepackage[noend]{algpseudocode}


\usepackage{enumitem,amssymb}
\newlist{todolist}{itemize}{2}
\setlist[todolist]{label=$\square$}
\newcommand{\norm}[1]{\left\lVert#1\right\rVert}
\newcommand{\Tr}[1]{\text{Tr}\left(#1\right) }
\newcommand{\vt}[1]{\text{vec}\left(#1\right) }
\newcommand*{\tran}{^{\mkern-1.5mu\mathsf{T}}}

\newcommand{\Logm}[1]{\text{Logm}\left(#1\right) }
\newcommand{\Expm}[1]{\text{Expm}\left(#1\right) }

%\modulolinenumbers[5]

\newcommand\blfootnote[1]{%
	\begingroup
	\renewcommand\thefootnote{}\footnote{#1}%
	\addtocounter{footnote}{-1}%
	\endgroup
}
\colorlet{shadecolor}{blue!20}
\newcommand{\mgw}[1]{\textcolor{orange}{#1}}
\newcommand{\vj}[1]{\textcolor{blue}{#1}}
\newcommand{\jxu}[2]{\textcolor{olive}{\sout{#1}} \textcolor{red}{#2}}

\usepackage[style=numeric, sorting=none, giveninits=true, backend=bibtex, maxbibnames=3, url=false, isbn=false, doi=false]{biblatex}
%style=numeric-comp ,
\bibliography{library}

\title{Checklist for personalized NIBS for L2 learning}

\makeatletter
\def\blx@maxline{77}
\makeatother


\nolinenumbers
\begin{document}
	
	%\begin{itemize}
	%	\item Riemannian methods suffer from poor interpretability and high computational cost
	%	\item Spatial filters from the tangent space enable interpretable and efficient computation
	%	\item Applying the filters are equivalent to the original formulation under mild conditions
	%	\item Proposed pipelines meet/increase the original method's accuracy in a meta-analysis
	%	\item We include open-source off-the-shelf code for the proposed framework
	%\end{itemize}
	
	\numberwithin{equation}{section}
	\numberwithin{figure}{section}
	\numberwithin{table}{section}
	
	\begin{center}
		{ \Large  
			\textbf{Checklist for personalized NIBS for L2 learning - Electrodes placement}
		}
	\end{center}
\vspace{1cm}

\begin{itemize}
	\item \textbf{Montage}
	
	\begin{todolist}
		\item TP7: Anode (Red) \& TP8: Cathode (Blue)
	\end{todolist}
	\item \textbf{Preparation tACS Electrodes}
	\begin{todolist}
		\item Measure brain size (circumference) $\Rightarrow$ Choose and put on EEG cap $\Rightarrow$ Adjust position (10-20 system) $\Rightarrow$ Fasten the chin strap
		\item Mark the position of tACS electrodes, i.e., TP7/8
		\item Confirm the marked area and remove the caps 
		\item Mark the circle area with some arc of diameter 36mm 
		\item Remove the hair in that area as much as possible
		\item Connect the wire but DO NOT turn the stimulator on.
		\item Apply a thin layer of Ten20 gel on the tACS electrodes (a sparse application is recommended) and place it in the correct location as marked area indicate.
		\item Turn on stimulator and monitor the impedance. In the meantime, carefully putting some pressure on the tACS electrodes while paying close attention to the location of tACS electrodes 
		\item Add more paste if necessary, but still, as less as possible. And make sure the electrodes are within the marked area.
		\item Continue putting pressure until the impedance is under 10k$\Omega$
		\item Remove the excess gel with a cotton swab
		\item Make sure the electrodes are firmly located on the scalp
		\item Double-check the impedance under 10k$\Omega$
		\item Turn off the stimulator
	\end{todolist}
	\item \textbf{Mounting the EEG cap}
		\begin{todolist}
			\item Carefully and gently put on the cap and MAKE SURE the tACS electrodes are not shifted and no Ten20 gel leaked out
			\item Fasten the strap of the EEG cap
		\end{todolist}
	
	\item \textbf{Preparation for EEG electrodes}
		\begin{todolist}
			\item Gel the ground and reference EEG electrodes first.
			\item Proceed with the EEG electrodes located in the vicinity of the tACS electrodes: Inject the gel with the needle tip pointing in a direction away from the tACS electrode and apply as LESS GEL as possible.
			\item Apply gel to the rest EEG electrodes (First round)
			\item In the second round, for the electrodes at the vicinity of tACS electrodes, use the cotton swab to increase contact between scalp and electrodes. DO NOT USE SYRINGE to do the second round for these electrodes.
			\item Use syringe do the second round gelling for all rest electrodes
			\item Check the impedance and thrifty gel the electrodes which do not show a green/yellow impedance.
			\item Test any bridging effect exist between tACS and EEG electrodes, i.e., apply 1mA sinusoidial signal and observe any saturation of EEG electrodes from recording signal.
			\item Check impedance again and start recording.
		\end{todolist}
\end{itemize}
	\printbibliography
	
	
	
	
	
\end{document}
